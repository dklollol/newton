\documentclass[a4paper,12pt]{article}

\title{REX Exam 2016}
\author{Group 6:\\Niels\\Troels\\Mark\\Kristian}

\usepackage[T1]{fontenc}
\usepackage{lmodern}
\usepackage[utf8]{inputenc}
\usepackage[british]{babel}
\usepackage{microtype}
\usepackage{underscore}
\usepackage{amsmath}
\usepackage{graphicx}
\usepackage[hidelinks]{hyperref}

\setlength{\parskip}{1ex}
\setlength{\parindent}{0pt}
\setlength{\parfillskip}{30pt plus 1 fil}

\begin{document}

\maketitle
\newpage
\section{Implementation}

To build the code, run \texttt{make}.  It builds the program \texttt{main}.  We
use OpenCV 2.4.

Our main loop works like this:

\begin{enumerate}
\item Get a frame from the webcam.
\item Detect landmarks by calling \texttt{cam.get_object}.
\item Compute particle weights.
\item Resample particles.
\item Move the robot according to our strategy.
\item Update particles according to how much the robot has moved.
\item Add uncertainty.
\item Estimate pose.
\item Draw and render for debugging purposes.
\end{enumerate}

We set the Scorpion robot to always drive with 20 cm per second, and always turn
with 20 degrees per second.


\subsection{Code Profiling}

We would like our solution to be as fast as possible, so we have timed the parts
of the main loop to see what might be improved.

Most of the steps complete in less than 20 milliseconds.  Only landmark
detection and robot moving are really time-consuming steps.  We do not dare to
change the handed out landmark detection code, and we cannot reliably make the
robot move any faster, so we are stuck with a main loop that spends at least 300
milliseconds in each iteration.


\subsection{Webcam Hack}

Since each iteration of the main loop is pretty slow, and since the webcam
updates pretty often, we don't use \emph{all} webcam frames, but only the newest
one whenever the code starts a new loop iteration.

However, OpenCV uses a FIFO buffer for grabbing webcam frames, so that you get
the first frame in the buffer and not necessarily the \emph{newest} frame.  The
solution, which is in the handed out code, is to set the property
\texttt{CV_CAP_PROP_BUFFERSIZE} to $1$, but this property does not work with all
webcams\footnote{\url{http://docs.opencv.org/2.4/modules/highgui/doc/reading_and_writing_images_and_video.html}},
including ours.

We had to comment out the \texttt{CV_CAP_PROP_BUFFERSIZE} fix, which resulted in
us getting outdated frames from our webcam, meaning we had to find another fix.

Our current fix works by starting a thread which continually grabs frames from
the camera and always stores the newest frame in a global variable, which the
main loop can then access.  We use a couple of synchronization primitives to
avoid race conditions.  It appears to work, although it would have been nicer
had our webcam supported the OpenCV property.


\section{Particle Filter}

We use the same particle filter as in our exercise 5, with a few additions:

\subsection{Adjustable Variance}

FIXME

Andet nyt?


\section{Driving Strategy}

We use a state-based approach for our driving strategy: We split the strategy
into multiple states, and let each state handle a small part.  The driving
strategy code is part of the main loop.

\begin{description}
\item[searching_random]\hfill\\
prut\end{description}



\section{Discussion}

testing, conclusion


\end{document}
