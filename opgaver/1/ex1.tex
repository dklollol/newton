\documentclass[a4paper,12pt]{article}

\title{Exercise 1: Getting Started - Driving}
\author{Gruppe 6}

\usepackage[T1]{fontenc}
\usepackage[utf8]{inputenc}
\usepackage[british]{babel}
\usepackage{microtype}

\usepackage{amsmath}
\usepackage{libertine}

\usepackage{graphicx}

\usepackage[hidelinks]{hyperref}

\setlength{\parskip}{1ex}
\setlength{\parindent}{0pt}
\setlength{\parfillskip}{30pt plus 1 fil}

\begin{document}

\maketitle

\section{ Sub-exercise 1: Simple movement}

For the first assigment we chose to use a constant speed for the scorpion at 0.2 which equates to 20 cm/sec or 1 meter in 5 seconds.

We also ran it in the simulator

Tried with speed 1 in the simulator, but probably too fast for real robot

\subsection{square}

Our program square loops driving 1 meter in 5 secounds,  and turns 90 degrees left in 6 seconds. This program is looped 4 times before stopping.

\section{ Sub-exercise 2: Continuous motion}

\subsection{simple-eight}

\subsection{trigonometry}

\end{document}
