\documentclass[a4paper,12pt]{article}

\title{Exercise 2: Obstacle Avoidance}
\author{Gruppe 6}

\usepackage[T1]{fontenc}
\usepackage[utf8]{inputenc}
\usepackage[british]{babel}
\usepackage{microtype}

\usepackage{amsmath}
\usepackage{libertine}

\usepackage{graphicx}

\usepackage[hidelinks]{hyperref}

\setlength{\parskip}{1ex}
\setlength{\parindent}{0pt}
\setlength{\parfillskip}{30pt plus 1 fil}

\begin{document}

\maketitle

\section{Measurements}

We started by measuring the robot's front IR sensor on seven distances: 0 cm, 10
cm, 20 cm, 30 cm, 40 cm, 50 cm, and 60 cm.  We positioned the robot in front of
a green piece of paper in a lit room and had it make 20 IR measurements at each
distance.  The distance was from the robot's bumper to the piece of paper.


\section{Driving}

\subsection{Issue: Pull mode vs. push mode}

We had a problem with wrong IR readings on the Scorpion robots \emph{and} in the
Stage simulator.  When not constantly calling the \texttt{Read()} function, the
\texttt{libplayer} API sometimes gave us old readings instead of current ones.
After doing some research, we found out that the issue was because of how the
default data mode was used by
\texttt{libplayer}\footnote{\url{http://playerstage.sourceforge.net/doc/Player-cvs/player/group__libplayerc__datamodes.html}}.
By default it uses PULL mode.  This mode sends messages to the client when asked
to, but if the client does not read often enough, buffer overflows in the server
queue can occur.  To get rid of this problem, and not have to call
\texttt{Read()} all the time, we have set a \texttt{libplayer} replace rule:

\begin{verbatim}
robot.SetReplaceRule(true, PLAYER_MSGTYPE_DATA, -1, -1);
\end{verbatim}

This rule tells \texttt{libplayer} to ignore the buffer queue and just replace
every old measurement with every new measurement, thus ensuring that
\texttt{Read()} gets us the newest measurement.


\subsection{Idea}

Mark skriver noget?

\end{document}
