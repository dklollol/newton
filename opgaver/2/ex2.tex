\documentclass[a4paper,12pt]{article}

\title{Exercise 2: Obstacle Avoidance}
\author{Gruppe 6}

\usepackage[T1]{fontenc}
\usepackage[utf8]{inputenc}
\usepackage[british]{babel}
\usepackage{microtype}

\usepackage{amsmath}
\usepackage{libertine}

\usepackage{graphicx}

\usepackage[hidelinks]{hyperref}

\setlength{\parskip}{1ex}
\setlength{\parindent}{0pt}
\setlength{\parfillskip}{30pt plus 1 fil}

\begin{document}

\maketitle

\section{Measurements}

We started by measuring the robot's front IR sensor on seven distances: 0 cm, 10
cm, 20 cm, 30 cm, 40 cm, 50 cm, and 60 cm.  We positioned the robot in front of
a green piece of paper in a lit room and had it make 20 IR measurements at each
distance.  The distance was from the robot's bumper to the piece of paper.


\section{Driving}

\subsection{Issue: Pull mode vs. push mode}

Niels skriver noget her.  Virke i Stage også.

Read()

\subsection{Idea}

Mark skriver noget?

\end{document}
