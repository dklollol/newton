\documentclass[a4paper,12pt]{article}

\title{Exercise 5: Self Localization}
\author{Group 6:\\Niels\\Troels\\Mark\\Kristian}

\usepackage[T1]{fontenc}
\usepackage{lmodern}
\usepackage[utf8]{inputenc}
\usepackage[british]{babel}
\usepackage{microtype}
\usepackage{underscore}
\usepackage{amsmath}
\usepackage{graphicx}
\usepackage[hidelinks]{hyperref}

\setlength{\parskip}{1ex}
\setlength{\parindent}{0pt}
\setlength{\parfillskip}{30pt plus 1 fil}

\begin{document}

\maketitle

\section{Implementation}

\subsection{Particle Filter}




\subsection{Driving Strategy}

We use a state-based approach for our driving strategy: We split the strategy
into multiple states, and let each state handle a small part.  The driving
strategy code is part of the main loop.

\begin{description}
\item[searching]\hfill\\
This is the start state.  The robot turns counter-clockwise until it recognizes
a landmark, after which it goes to state \textbf{align}.

\item[align]\hfill\\
The robot aligns itself to point directly at the recognized landmark (within
a low threshold).  Once aligned, it goes to state \textbf{approach}.

\item[approach]\hfill\\
We don't want the robot to be too far away from the landmark.  If the robot is
more than 150 cm from the landmark, the robot approaches it.  Once it's at most
150 cm away, it goes to state \textbf{drive_around_landmark}.

If the robot is not directly in front of the landmark, it goes back to the state
\textbf{align}.

\item[drive_around_landmark]\hfill\\
At this point the robot has seen the first landmark, and we now want it to find
the second landmark.  The robot drives around the first landmark in a square and
keeps checking for the second landmark.  This should explore all angles.  Once
it finds the second landmark, it goes to state \textbf{triangulating}.

\item[triangulating]\hfill\\
We run the particle filter a few times.  The robot then goes to state
\textbf{drive_to_center}.

\item[drive_to_center]\hfill\\
The robot has now seen both landmarks, and we choose to trust its estimated
coordinates.  The robot drives towards $(150, 0)$ and then goes to state
\textbf{arrived_at_center}.

\item[arrived_at_center]\hfill\\
Success!
\end{description}


\end{document}
